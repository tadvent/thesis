% !Mode:: "TeX:UTF-8"

\chapter{调度程序输入格式}

程序的输入格式如下:

第1行整数 N 表示待调度的总任务数。

接下来 N 行,每行5个整数 C T O D S 分别表示每个任务的最坏执行时间、释放周期、首次释放时间偏移、相对时间限以及任务是否是自依赖的。其中 T 或 D $\leqslant$ 0 则认为该任务非周期任务。

下一行一个整数 K 表示任务间数据通信关系的个数

接下来 K 行,每行5个整数 Tp p Tc c Dpc 分别表示从任务 Tp 产生 p 个数据发送给任务 Tc,每次读入 c 个数据,开始时的 Delay 是 Dpc。

下一行一个整数 J 表示周期倍数因子

下一行一个整数 M 表示目标平台的处理器数

下一行一个整数 s 表示核间通信速率(单位时间传输的数据量)

下一行一个整数 L 表示核间连接数

下面 L 行,每行两个整数 p1 p2 表示处理器 p1 与 p2 相连
