% !Mode:: "TeX:UTF-8"
\chapter{绪论}

% \emph{TODO: 调整格式、补全引用}

\section{课题研究背景}

% -------------------------------
\subsection{多核分区系统}
在航空电子等方面,传统的联合式结构中,每个单独的计算机系统只负责某一项具体的应用功能,所有功能模块通过特定的总线接口与相关协议有机的联系起来,控制整机的正常运行。这种结构的好处是他们内在的错误具有隔离性,即一般一个功能模块的失效并不会影响其他功能的正常运行。但同时其缺点是每个功能都需要自己独占的计算机,系统中处理器的数量会随着系统功能的丰富而增加,造成整个系统在空间、电力、重量、冷却、安装、维护等方面的重复负担,增加了维护成本、限制了系统规模。

为了提高系统的集成度,增加计算机资源的利用率,人们提出了系统综合化的要求。在综合化的系统中,一个计算机系统能够为多个应用提供计算资源,从而提高系统的利用效率。但综合化系统会破坏原来硬件的天然隔离,增加错误传播的几率。为了解决这个问题,分区系统被提了出来。使用分区机制可以为运行在系统中的不同应用建立独立分区,在时间和空间两个维度提供资源,使应用之间相互隔离,保持联合式结构中的优点。

% 可以添加 综述 里关于多核发展的描述 \cite{SurveyHRT}
另一方面,随着集成电路工艺的发展遇到瓶颈,单核处理器的性能已经达到极限,单纯的通过提高主频等传统方式已经无法显著提高处理器的性能。通过多核、众核架构来更好的并行处理任务成为了新的趋势,主要处理器芯片商也纷纷推出多核产品,目前已在通用领域得到广泛应用。因此,分区系统向多核处理器平台发展也将成为不可避免的趋势。


\subsection{实时调度算法}


目前,关于多核实时系统的调度算法研究,多集中于任务的时间约束,以及提高处理器的利用率等方面\upcite{SurveyHRT}。多核的并行任务分配大部分是NP完全问题\upcite{SchOSRT},这给算法设计带来了难度。此外,多核实时调度算法针对任务间数据流通、协作而产生的偏序依赖关系的情况还缺少成熟的结论。

随着系统规模的扩大,实时任务间的关系也日趋复杂。如文章\cite{Without}中提到的情况,在构建一个关键的嵌入式控制系统时,系统中的控制循环包括传感器、控制算法和调节系统状态的传动部件等。这类系统的构建通常需要多人团队并行的分工合作,分别开发系统中的不同部分,之后再将这些部分(或者称作任务)组合成一个系统。这类集合工作经常需要人工在离线情况下对任务排序,来保证任务的执行是可预测、可确定的。但在目前成熟可靠的嵌入式操作系统中,如用于汽车工业的 OSEK\upcite{OSEK}系统、面向宇航的 RTEMS\upcite{RTEMS} 系统、以及遵循 ARINC 653 标准\upcite{ARINC653} 的航电系统(如 VxWorks 653\upcite{VxWorksRTCA})等,对带有数据依赖的任务并没有提供直接支持。这意味着任务间通信的确定性保证通常需要手工先对任务部分排序,然后再交由操作系统调度,而这个过程是冗繁的。

%再举个例子

%\section{课题来源与意义} %=========================================

\section{论文研究目标与内容}


结合以上几方面的发展趋势及多核调度算法在实际问题中所遇到的挑战,本文以多核分区系统平台的任务为模型,提出一种面向通信的调度算法,综合考虑任务的周期性时间约束和任务间由于数据通信所产生的数据依赖约束,产生满足以上约束条件的静态调度。

本文针对多核分区的实时系统,研究它的调度算法。主要研究目标包括:
\begin{enumerate}
\item	调研现有多核或实时调度算法,讨论在多核分区实时系统中,这些算法在调度包含周期性时间约束和数据通信产生的数据依赖约束的任务时所遇到的困难。包括如何解决周期任务间的依赖关系、保证实时任务的时间限制、考虑不同处理器间的通信延迟等。
\item	针对以上问题抽象出一个问题模型,在该模型上实现一种多核分区实时调度算法,在给定一个周期性任务集和他们的时间限制,并给出任务间的数据依赖关系的情况下,在目标多核分区处理器平台寻找满足以上约束条件的静态调度。在能够调度时给出考虑核间通信延迟下的一种可行调度方案。
\item	验证算法的正确性,分析算法的时空复杂度,并结合例子分析算法对通信约束的处理。
\end{enumerate}


本论文的主要研究内容有以下几个方面:
\begin{enumerate}
\item	调研多核系统现有的并行调度算法及其优势和局限性。
\item	设计调度算法,以满足对多核分区系统中有依赖关系的实时任务的有效调度。主要考虑以下问题:
  \begin{itemize}
    \item	任务的实时性要求。如周期性、时间限等。
    \item	任务的数据依赖关系约束。如任务C的执行需要任务A 和B 的输出结果时,不能使C 先于任务A 和B 执行。
    \item	核间的通信开销。当A向其他核上执行的任务B发送数据时,B要在数据到达后才能开始执行;如果A 和B 在同一处理器时不存在此延迟,或延迟可忽略。
    %\item	在给定的多核架构上充分发掘任务间的并行性,以使调度结果尽量接近最优。
  \end{itemize}
\item	从理论上证明算法对实时任务在时间上的保证。
\item	构建调度算法模拟程序,并验证调度的任务满足时间与数据约束。%并实测算法的运行效率和调度结果的优劣。
\end{enumerate}

\section{国内外研究现状}

%介绍实时任务的两方面:传统周期性约束 以及 DSP 中的数据约束
实时任务的正确性不仅要求任务运行结果在逻辑上正确,还要求任务必须在一定的时间限内运行结束。从周期性和偶发任务模型来说,实时任务包括周期性任务、偶发性任务和非周期任务等,它们的特征是任务均以时间为驱动,即任务什么时候释放、是么时间段可以执行以及任务什么时候必须执行完毕都是由时间来决定的;与此相对的,DSP 系统中的数据流程序则属于另一类的实时任务,它们以数据驱动为特征,即什么时候可以执行,哪个任务先执行,都是由任务间的数据流动来决定的。

%分两段分别介绍时间约束的实时系统,如 VxWorks 等 和 DSP 中的数据约束 各自的处理方法,如 时间系的调度为何不满足需求、SDF 系的调度为何也不满足需求
周期性和偶发任务模型是大多数传统的实时操作系统所支持的任务模型。在绝大多数发表的关于周期性和偶发任务模型的调度研究中,都假定任务间是相互独立的\upcite{SurveyHRT},即不考虑因处理器以外的资源争用造成任务阻塞。大部分多核的实时任务调度问题都是NP完全的,因此一些启发式算法被提了出来,如 Thakor, D. 和 Shah, A. 提出的D\_EDF 算法\upcite{DEDF},Zhu Xiangbin 和 Tu Shiliang 提出的基于 myopic 算法的启发式算法\upcite{Zhu2003} 等。但由于受任务模型所限,这些算法均未能考虑任务间制约关系的问题。 % 参考那篇大综述,有一章讲了争用
此外,Rajkumar 等人在\cite{Contention}中使用固定优先级的方法在多核分区系统中实现了类似单处理器调度中用于解决共享资源争用的 PCP (Priority Ceiling Protocol) 策略\upcite{SContention},一定程度上打破了周期性和偶发任务模型对于任务相互独立的限制。但这类方法用于解决多核系统中多任务互斥访问共享资源的调度问题,对于任务间数据通信所产生的先后依赖关系仍未涉及。


在数据驱动的任务模型中,由于DSP系统对大数据量吞吐率的需求,任务多采用静态调度,它的并行调度算法也较好的解决了任务间的数据传输、依赖关系等问题。设计流式程序所要面对的一个挑战是如何有效表达程序中的并行性\upcite{Challenge}。而基于计算模型 (Model-of-Computation, MoC) 的设计则成为了解决这一挑战的事实标准\upcite{MoCdefacto}。% 提出 SDF 方法
如EDWARD ASHFORD LEE 等人提出的SDF 方法\upcite{SDF1987,SDF_C} 即是一种在 DSP 中使用广泛的计算模型,它能够较好的表述任务间的数据传输、依赖关系等,并由此得到有效的静态调度表。 但由于DSP 系统是数据驱动型的,这点从本质上与时间驱动型的实时系统有所不同,DSP 中的一些调度算法(包括 SDF 相关的方法)并不能直接用来解决带有周期性时间属性的实时任务调度问题。在解决时间约束方面Margarete Sackmann 等人提出的一个考虑时间和能耗约束的快速启发式算法\upcite{FastHeu} 在解决单个任务的时间限制约束上有一定进展,但仅将时间限制作为调度的参考,没有给出理论上的严格保证,也未考虑任务的周期性约束条件等。

% 再一段介绍有些方法企图将两者结合起来,如。。。等
% 主要是那几篇解决问题比较类似的论文
也有些文章综合考虑了任务的时间约束以及任务间的偏序依赖关系。如\cite{Xujia1993}提出了这样的任务模型:有一个任务集,其中每个任务有确定的释放时间和结束时间限,每个任务又由一系列包含偏序关系和互斥关系的子任务组成。文中针对此任务模型提出了在多核平台的静态最优调度算法。虽然该模型同时考虑了任务的时间以及依赖约束,但其时间约束仅针对整个任务,对子任务没有单独的限制。此外,与本文的工作相比,\cite{Xujia1993} 中的模型也有几点不同:1)~时间约束是一次性的,未考虑周期性调度的情况;2)~未考虑任务或子任务之间通信延迟的影响。
另外,文章\cite{Without}提出并研究了关键嵌入式系统中,带有先后关系约束的周期性任务集的调度问题。文中将任务的先后关系分为两类来解决:(1)~简单先后关系:带有先后关系的任务具有相同的周期;(2)~扩展的先后关系:带有先后关系的任务可以有不同的周期。对于简单先后关系的任务,文中将任务在周期内按不同的时间偏移重新组织,保证在周期内满足任务间先后关系的约束;而对于扩展的先后关系,文中取两个周期的最小公倍数周期作为一个大周期,列举大周期内各任务多次执行之间的所有先后关系约束,将其化为相同周期的任务处理。之后,由于每个处理器上的调度仅采用固定优先级的传统周期任务实时调度算法,该策略没有能够保证任务之间先后关系的机制,因此需要单独调整各个带有先后关系的任务的周期属性,推后首次释放时间、缩减运行时间限来保证任务间的先后关系约束。这样造成的一个明显缺陷是,当任务中先后关系较多时,原本有可能调度的任务由于推后了首次释放时间和减少了时间限造成无法调度的结果。另外,本文也没有链路上的通信调度。
文章\cite{HRTSCSDF}将非环的 CSDF 图(一种类似 SDF 的计算模型)转化为周期性任务,试图在转化后仅用传统的针对相互独立任务的周期性调度算法来调度这些任务,同时还保证任务间满足 CSDF 图所述的数据依赖关系。该文的方法与\cite{Without}类似,都是希望将数据依赖关系转化为独立的周期性任务,通过调整任务的周期属性来保证任务间的先后依赖关系。由于转化为周期性任务限制了任务调度的灵活性,因此该方法并不保证对所有数据依赖都能成功转化。与之相比,本文的方法将周期性的时间约束关系转化为数据依赖关系,并与原任务间的数据依赖共同构成统一的 GSDF 图。由于上述转化过程是不损失调度灵活性的,因此该转化过程对任务的可调度性无损害。

% 中期检查

在分布式计算方面,人们对多处理器的并行调度也有较多研究,由于运算环境的复杂性,各处理器间的通信效率、时延以及整个任务集的总调度长度等方面是主要的关注点,如文章\cite{DLS} 提出的Mapping Heuristic 算法,文章\cite{Leinberger2000,Yu1991} 提出的DLS 算法等,考虑了数据通信的消息调度等问题。但同样这些算法也缺少对周期性实时性任务的支持,并不能从理论上保证任务的时间需求。


\section{论文研究成果}
%直接照搬 abstract,待修改

针对多核实时系统上的任务调度问题,本文提出一种面向通信的静态调度 (Communication-Oriented Static Scheduling, COSS) 算法。该算法能够在考虑任务间数据通信和周期任务时间限的情况下,在运行前将任务静态调度到多核处理器平台上,同时保持任务之间的通信关系和时间限制得到满足。在有效调度能够被找到的情况下,算法将会给出各个核上的调度序列,以及核之间链路上的消息调度序列。

\section{本文的组织}

本文其余部分组织如下:

第二章首先简要介绍目前主要多核实时调度算法的分类、各自的特点。其次论文讨论相关的几种多核平台的计算模型极其特点,所采用相关算法的考虑因素。第三节介绍了 DAG 调度的主要几类算法,并分析了各自的特点。第二章最后对多核分区操作系统的特点做了简要介绍。

% 直接说三四两章详细介绍算法设计,不用这么具体写每章是什么
本文三、四、五章分别详细介绍了 COSS 算法的三个步骤的具体设计及思路,给出了每个过程中所要解决的问题,以及所选取的具体方法的原因,提出了每个过程的具体算法过程,并分析了算法复杂度。

第六章根据前述算法流程的设计,介绍算法的具体实现过程,模块划分以及各模块在具体实现中遇到并解决的一些问题。

第七章从几个例子验证了调度算法对任务调度结果的正确性,并结合相同任务集在不同连接平台的调度结果分析了处理器间的连接对任务分配的影响。