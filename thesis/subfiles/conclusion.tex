% !Mode:: "TeX:UTF-8"

\chapter*{结论与展望}
\addcontentsline{toc}{chapter}{结论与展望}

\section*{总结论文内容}
本文对包含数据通信而产生数据依赖关系的一组周期性实时任务调度问题提出了针对多核分区系统平台的静态调度算法,解决了任务间同时存在时间约束与数据约束的问题,在进行任务调度的同时,也生成了任务在处理器之间链路上的消息调度。

本文的主要工作内容包括以下几点:

首先,本文针对待解决的调度问题调研了现有的周期性与偶发任务模型和数据驱动的任务模型两类实时任务模型的多核调度算法。分析了现有方法在调度同时包含周期性时间约束和数据通信约束的任务集时的不足之处,并比较了各算法所解决的问题范围以及各自的优劣。

其次,本文在已有算法的基础上设计了面向通信的多核分区静态调度算法,为同时包含通信关系造成的数据依赖以及周期性时间约束任务的调度问题提出了一种解决方案。本文对算法的正确性给出了证明,并提出了处理器排序等优化对调度结果的影响。

最后,本文设计并实现了算法模拟程序,并结合例子验证了算法流程、算法自依赖约束以及处理器排序等因素对调度结果的影响。

\section*{下一步工作展望}

处理跨周期任务及跨周期消息的调度问题,以解决任务时间限大于释放周期的周期性任务的调度,以及某些情况下由于周期内排在后面的任务与下周期任务间消息传递过久,使得消息调度无法在本周期结束,导致的无法调度情况。

研究寻找更好的消息路由方法。最短路有可能造成消息在枢纽节点周围形成拥塞,而不能有效利用外围链路。

虽然由数据依赖造成的优先关系可用SDF描述,但还有一些任务间的优先关系无法用SDF 描述。需要进一步界定用虚拟数据结合SDF可表示的任务间优先关系的集合。
