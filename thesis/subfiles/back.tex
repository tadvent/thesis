% !Mode:: "TeX:UTF-8"

\chapter{攻读硕士学位期间取得的学术成果}
%% 此处标题及内容请自行更改
\noindent 发表论文:

\begin{enumerate}
  \item Jinlin Wang. Scheduling of Periodic Tasks with Data Dependency on Multiprocessors[A]. {\em International Conference on Computer Science and Communication Technology}[C], 青岛, 2012. 已录用
\end{enumerate}



%\noindent 发表论文:
%\begin{enumerate}
%
%\item
%Gang Bai and Yue Qi. An Interactive 3D Exhibition System with
% Global Illumination for Digital Museum. In Lecture Notes in
% Computer Science, 2009, Volume 5670, Learning by Playing.
%Game-based Education System Design and Development, Pages 85-92.
%
%\item
%Hu Yong, Qi Yue and Bai Gang. Modeling and Editing Isotropic BRDF.
%In proceedings of the Second International Conference on Modeling,
%Simulation and Visualization Methods (WMSVM). 15-16 May, 2010,
%Sanya, China. Pages 74-77.
%
%\end{enumerate}


%\noindent 申请专利:
%\begin{enumerate}
%
%\item
%齐越,马宗泉,白刚.基于任意位置多球的光源方向标定[P]. 中国发明专利(200910092909), 公开日2010年2 月17 日
%
%\end{enumerate}


\chapter{致谢}

时光荏苒,我的研究生生活即将结束。在这两年多中的时光中,我经历了很多,有过欢笑,有过收获,有过困难,也有过挫折。最重要的是,我从周围的老师、同学、朋友身上学到了很多,有知识方面的积累,也有生活阅历的充实。在此,我要感谢各位老师、同学和家人的关心和帮助。

首先,我要感谢我的导师龙翔教授。龙老师在学科方向上的洞察力和清晰准确的思路让我非常敬佩,与龙老师的讨论使我能够及时发现研究内容的重点,指出我的问题所在,使我收获良多,思路豁然开朗。龙老师严谨治学的态度和对学生悉心关怀使我在学习和生活上都有很大收获。

感谢姜博老师对我的研究内容和论文提出的指导和建议,使我对论文的组织和内容安排有了更清晰的认识。

感谢杨经纬博士在我的研究和准备毕设期间一直对我提供帮助和鼓励,对我研究的内容提出了很多启发性的建议,同时也鞭策我不断努力奋斗。

感谢实验室的所有同学,我们一同生活、一同学习。感谢陈龙、刘乃溪、韩楠、史彪、端文宗,我们一起玩闹、一起吃饭、一起K歌、一起为了共同的目标奋斗,我们互相帮助,使我的研究生生活充实而丰富多彩。

还要感谢家人的养育之恩,感谢他们在我读研期间的关怀和支持,不论遇到任何困难,他们都给我提供最坚实的支持与鼓励,让我能够向着自己的目标不断前进。
